\documentclass[a4paper, 12pt]{article}
\usepackage[utf8]{inputenc}
\usepackage[utf8]{vietnam}
\usepackage{amsmath}
\usepackage{amsfonts}
\usepackage{amssymb}
\usepackage{type1cm}
\usepackage{comment}
\usepackage{enumerate}
\usepackage{tikz}
\usetikzlibrary{positioning}
\usepackage[left=1cm,right=1cm,top=2cm,bottom=2cm]{geometry}
\everymath{\displaystyle}
\newcommand{\pfm}[1]{\left({#1}\right)}
\newcommand{\unit}[1]{~#1}
\begin{document}
\begin{center}
\begin{small}
\begin{tabular}{cc}
TRƯỜNG ĐẠI HỌC& \textbf{CỘNG HÒA XÃ HỘI CHỦ NGHĨA VIỆT NAM}\\
KỸ THUẬT -- CÔNG NGHỆ CẦN THƠ&  \textbf{Độc lập -- Tự do -- Hạnh phúc}\\
\textbf{KHOA ĐIỆN -- ĐIỆN TỬ -- VIỄN THÔNG} & \\
\textbf{BỘ MÔM ĐIỆN -- ĐIỆN TỬ} & \emph{Cần Thơ, ngày 10 tháng 10 năm 2016} \\
\end{tabular}
\end{small}
\end{center}
\vspace{.7cm}
\begin{center}
	\begin{Large}
	\textbf{ĐỀ THI GIỮA KỲ} \vspace{.3cm}\\
	\textbf{MÔN HỌC THIẾT KẾ HỆ THỐNG ĐIỆN}\\
	\tikz{\draw[thick] (0,0) -- (5,0);}
	\end{Large}
	\vspace{1cm}
	\begin{center}
	\begin{tabular}{cl}
		-- & Thời gian làm bài: \emph{90 phút (sinh viên được phép sử dụng tài liệu)}.\\
		-- & Hình thức làm bài: Tự luận. \\
		-- & Nội dung đề thi gồm có 01 câu hỏi thể hiện trên 01 trang.\\
	\end{tabular}
	\end{center}
\end{center}
\section*{\hspace{1cm}Đề bài}
\begin{itemize}
\item[] Cho đường dây ba pha có chiều dài $210 \unit{km}$, điện áp đầu nhận là $U_R = 220 \unit{kV}$, với tần số $50 \unit{Hz}$ chuyển cho đầu nhận công suất $P_R = 150 \unit{MW}$, cho hệ số  công suất ở đầu nhận là $\cos_R = 0.83$ trễ. Điện trở trên mỗi pha $r_0 = 0.16 \unit{\Omega/km}$, cảm kháng trên mỗi pha $x_0 = 0.8 \unit{\Omega/km}$ và điện dẫn trên mỗi pha $b_0 = 10^{-6} \unit{\Omega^{-1}/km}$.
\begin{enumerate}[\it 1.]
	\item \textit{Dùng sơ đồ thay thế hình $T$ để tính các thông số sau:}\label{Diagram:hinh-T}
	\begin{enumerate}[a.]
		\item Vẽ sơ đồ thay thế hình $T$.
		\item Các thông số của đường dây.
		\item Điện áp và dòng điện đầu gửi.
		\item Công suất và hệ số công suất đầu gửi.
		\item Góc lệch pha giữa đầu gửi và đầu nhận $\Delta \varphi$.
		\item Độ sụt áp $\Delta U\%$.
		\item Tổn thất trên đường dây $\Delta P$.
		\item Hiệu suất truyền tải $H$.
		\item Vẽ giản đồ vector.
	\end{enumerate}
	\item \textit{Tính lại câu \ref{Diagram:hinh-T} với sơ đồ thay thế hình $\Pi$}.
\end{enumerate}
\end{itemize}
\subsection*{\hspace{2cm} DUYỆT BỘ MÔN \hspace{5cm} GIÁO VIÊN RA ĐỀ}
\vspace{1.5cm}
\begin{large}
\hspace{13cm}\textbf{Võ Minh Thiện}
\end{large}
\end{document}